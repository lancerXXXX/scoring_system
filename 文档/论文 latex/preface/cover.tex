% !TEX TS-program = XeLaTeX
% !TEX encoding = UTF-8 Unicode

%%%%%%%%%%%%%%%%%%%%%%%%%%%%%%%%%%%%%%%%%%%%%%%%%%%%%%%%%%%%%%%%%%%%%% 
% 
%	大连理工大学博士论文 XeLaTeX 模版 —— 格式文件 format.tex
%	版本:0.1beta1
%	最后更新:2018.5.23
%	修改者:王天宇 (E-mail: heavenlykingwty@gmail.com)
%	编译环境:macOS MacTex XeLatex (Texpad)
% 
%%%%%%%%%%%%%%%%%%%%%%%%%%%%%%%%%%%%%%%%%%%%%%%%%%%%%%%%%%%%%%%%%%%%%% 


\cdegree{\makebox[13.25cm][s]{大连理工大学本科毕业设计(论文)}}
\ctitle{大连理工大学本科毕业设计(论文)题目}
\etitle{The Subject of Undergraduate Graduation Project (Thesis) of DUT}

% 根据需要添加字符间距
\department{\makebox[6.1cm][c]{软件学院}}   
\csubject{\makebox[6.1cm][c]{软件工程}}
\cauthor{\makebox[6.1cm][c]{李某}}
\cauthorno{\makebox[6.1cm][c]{201990000}}
\csupervisor{\makebox[6.1cm][c]{}}
\teacher{\makebox[6.1cm][c]{}}
\cdate{\makebox[6.1cm][c]{\today}}

% 这里默认使用最后编译的时间,也可自行给定日期,注意汉字和数字之间的空格。
%\cdate{{\quad\;}\the\year~年~\the\month~月~\the\day~日}

\cabstract{
“摘要”是摘要部分的标题,不可省略。

标题“摘要”选用模板中的样式所定义的“标题1”,再居中;或者手动设置成字体:黑体,居中,字号:小三,1.5倍行距,段后11磅,段前为0。

摘要是毕业设计(论文)的缩影,文字要简练、明确。内容要包括目的、方法、结果和结论。单位采用国际标准计量单位制,除特别情况外,数字一律用阿拉伯数码。文中不允许出现插图。重要的表格可以写入。

摘要正文选用模板中的样式所定义的“正文”,每段落首行缩进2个汉字;或者手动设置成每段落首行缩进2个汉字,字体:宋体,字号:小四,行距:多倍行距 1.25,间距:段前、段后均为0行,取消网格对齐选项。

摘要篇幅以一页为限,字数限500字以内。

摘要正文后,列出3-5个关键词。“关键词:”是关键词部分的引导,不可省略。关键词请尽量用《汉语主题词表》等词表提供的规范词。

关键词与摘要之间空一行。关键词词间用分号间隔,末尾不加标点,3-5个;黑体,小四,加粗。关键词整体字数限制在一行。
}

\ckeywords{写作规范;排版格式;毕业设计(论文)}

\eabstract{
外文摘要要求用英文书写,内容应与“中文摘要”对应。使用第三人称,最好采用现在时态编写。

“Abstract”不可省略。标题“Abstract”选用模板中的样式所定义的“标题1”,再居中;或者手动设置成字体:Times New Roman,居中,字号:小三,多倍行距1.5倍行距,段后11磅,段前为0行。

标题“Abstract”上方是论文的英文题目,字体:Times New Roman,居中,字号:小三,行距:多倍行距 1.25,间距:段前、段后均为0行,取消网格对齐选项。

Abstract正文选用设置成每段落首行缩进2字,字体:Times New Roman,字号:小四,行距:多倍行距 1.25,间距:段前、段后均为0行,取消网格对齐选项。

Key words与摘要正文之间空一行。Key words与中文“关键词”一致。词间用分号间隔,末尾不加标点,3-5个;Times New Roman,小四,加粗。
}

\ekeywords{Write Criterion;Typeset Format;Graduation Project(Thesis)}
\makecover 