% !TEX TS-program = XeLaTeX
% !TEX encoding = UTF-8 Unicode

%%%%%%%%%%%%%%%%%%%%%%%%%%%%%%%%%%%%%%%%%%%%%%%%%%%%%%%%%%%%%%%%%%%%%% 
% 
%	大连理工大学博士论文 XeLaTeX 模版 —— 格式文件 format.tex
%	版本:0.71
%	最后更新:2010.12.22
%	修改者:Yuri (E-mail: yuri_1985@163.com)
%	编译环境:Ubuntu 10.04 + TeXLive 2010 + TeXworks
% Windows XP SP3 + CTeXLive 2009 + WinEdt 5.6
% 
%%%%%%%%%%%%%%%%%%%%%%%%%%%%%%%%%%%%%%%%%%%%%%%%%%%%%%%%%%%%%%%%%%%%%% 

%%%%%%%%%%%%%%%%%%%%%%%%%%%%%%%%%%%%%%%%%%%%%%%%%%%%%%%%%%%%%%%%%%%%%% 
% 页面设置
%%%%%%%%%%%%%%%%%%%%%%%%%%%%%%%%%%%%%%%%%%%%%%%%%%%%%%%%%%%%%%%%%%%%%% 
%\geometry{paperwidth=21.0cm,paperheight=29.7cm,inner=2.5cm,outer=2.5cm,bottom=2.5cm,top=3.5cm,footskip=0cm,headsep=0.9cm,headheight=0.5cm}
% A4 纸张
\setlength{\paperwidth}{21.0cm}
\setlength{\paperheight}{29.7cm}
% 设置正文尺寸大小
\setlength{\textwidth}{16.1cm}
\setlength{\textheight}{22.2cm}
% 设置正文区在正中间
\newlength \mymargin
\setlength{\mymargin}{(\paperwidth-\textwidth)/2}
\setlength{\oddsidemargin}{(\mymargin)-1in}
\setlength{\evensidemargin}{(\mymargin)-1in}
% 设置正文区偏移量,奇数页向右偏,偶数页向左偏
%\newlength \myshift
%\setlength{\myshift}{0.35cm}	% 双面打印的奇偶页偏移值,可根据需要修改,建议小于 0.5cm
%\addtolength{\oddsidemargin}{\myshift}
%\addtolength{\evensidemargin}{-\myshift}
% 页眉页脚相关距离设置
\setlength{\topmargin}{-0.05cm}
\setlength{\headheight}{0.50cm}
\setlength{\headsep}{0.90cm}
\setlength{\footskip}{1.47cm}
% 公式的精调
\allowdisplaybreaks[4]  % 可以让公式在排不下的时候分页排,这可避免页面有大段空白。

% 下面这组命令使浮动对象的缺省值稍微宽松一点,从而防止幅度
% 对象占据过多的文本页面,也可以防止在很大空白的浮动页上放置很小的图形。
\renewcommand{\topfraction}{0.9999999}
\renewcommand{\textfraction}{0.0000001}
\renewcommand{\floatpagefraction}{0.9999}

%%%%%%%%%%%%%%%%%%%%%%%%%%%%%%%%%%%%%%%%%%%%%%%%%%%%%%%%%%%%%%%%%%%%%% 
% 字体字号定义
%%%%%%%%%%%%%%%%%%%%%%%%%%%%%%%%%%%%%%%%%%%%%%%%%%%%%%%%%%%%%%%%%%%%%% 
% 字号
\newcommand{\xiaoer}{\fontsize{18pt}{23.46pt}\selectfont} %小二,按照word一倍行距为基线行距
\newcommand{\erhao}{\fontsize{22pt}{28.6pt}\selectfont} %二号,按照word一倍行距为基线行距
\newcommand{\sanhao}{\fontsize{16pt}{20.8pt}\selectfont} %三号,按照word一倍行距为基线行距
\newcommand{\xiaosi}{\fontsize{12pt}{15.6pt}\selectfont} %小四,按照word一倍行距为基线行距
\newcommand{\wuhao}{\fontsize{10.5pt}{13.65pt}\selectfont}%五号,原理同上,word的基线行距为字号的1.3倍
\newcommand{\xiaosan}{\fontsize{15pt}{19.5pt}\selectfont}     % 小三
\newcommand{\sihao}{\fontsize{14pt}{18.2pt}\selectfont}     % 四号
\newcommand{\yihao}{\fontsize{26pt}{33.8pt}\selectfont}	    % 一号
\newcommand{\xiaoyi}{\fontsize{24pt}{31.2pt}\selectfont}      % 小一
\newcommand{\xiaowu}{\fontsize{9pt}{11.7pt}\selectfont}	    % 小五
\newcommand{\song}{\CJKfamily{song}}
\newcommand{\hei}{\CJKfamily{hei}}
\newcommand{\kai}{\CJKfamily{kai}}
\newcommand{\xhei}{\CJKfamily{xhei}}

% defaultfont 默认字体命令
\def\defaultfont{\renewcommand{\baselinestretch}{1.25}\song
  \fontsize{12pt}{15.6pt}\selectfont}

% 设置目录字体和行间距
\def\defaultmenufont{\renewcommand{\baselinestretch}{1.25}\song
  \fontsize{12pt}{15.6pt}\selectfont}

% 固定距离内容填入及下划线
\makeatletter
\newcommand\fixeddistanceleft[2][1cm]{{\hb@xt@ #1{#2\hss}}}
\newcommand\fixeddistancecenter[2][1cm]{{\hb@xt@ #1{\hss#2\hss}}}
\newcommand\fixeddistanceright[2][1cm]{{\hb@xt@ #1{\hss#2}}}
\newcommand\fixedunderlineleft[2][1cm]{\underline{\hb@xt@ #1{#2\hss}}}
\newcommand\fixedunderlinecenter[2][1cm]{\underline{\hb@xt@ #1{\hss#2\hss}}}
\newcommand\fixedunderlineright[2][1cm]{\underline{\hb@xt@ #1{\hss#2}}}
\makeatother

%%%%%%%%%%%%%%%%%%%%%%%%%%%%%%%%%%%%%%%%%%%%%%%%%%%%%%%%%%%%%%%%%%%%%% 
% 标题环境相关
%%%%%%%%%%%%%%%%%%%%%%%%%%%%%%%%%%%%%%%%%%%%%%%%%%%%%%%%%%%%%%%%%%%%%% 
% 定义、定理等环境
\theoremstyle{plain}
\theoremheaderfont{\hei\bf}
\theorembodyfont{\song\rmfamily}
\newtheorem{definition}{\hei 定义}[chapter]
\newtheorem{example}{\hei 例}[chapter]
\newtheorem{algorithm}{\hei 算法}[chapter]
\newtheorem{theorem}{\hei 定理}[chapter]
\newtheorem{axiom}{\hei 公理}[chapter]
\newtheorem{proposition}[theorem]{\hei 命题}
\newtheorem{property}{\hei 性质}
\newtheorem{lemma}[theorem]{\hei 引理}
\newtheorem{corollary}{\hei 推论}[chapter]
\newtheorem{remark}{\hei 注解}[chapter]
\newenvironment{proof}{\hei{证明} }{\hfill $\square$ \vskip 4mm}

% 目录标题
\renewcommand\contentsname{\hfill 目  录 \hfill\vspace{-0.7cm}}
\renewcommand\listfigurename{\hfill 插~图~目~录 \hfill}
\renewcommand\listtablename{\hfill 表~格~目~录 \hfill}
\renewcommand{\bibname}{\hfill 参~考~文~献 \hfill}

%%%%%%%%%%%%%%%%%%%%%%%%%%%%%%%%%%%%%%%%%%%%%%%%%%%%%%%%%%%%%%%%%%%%%% 
% 段落章节相关
%%%%%%%%%%%%%%%%%%%%%%%%%%%%%%%%%%%%%%%%%%%%%%%%%%%%%%%%%%%%%%%%%%%%%% 
\setcounter{secnumdepth}{3}
\setcounter{tocdepth}{3}
% 设置章、节、小节、小小节的间距
%\titlespacing的最后一个参数是垂直间距,待修改
% \titleformat{\chapter}[hang]{\normalfont\xiaosan\hei\sf}{\xiaosan\hei\thechapter}{1em}{}{}
\titleformat{\chapter}[hang]{\song\xiaosi\hei\sf}{\xiaosi\hei\thechapter}{1em}{}{}
\titlespacing{\chapter}{0pt}{-3ex  plus .1ex minus .2ex}{18pt}
% \titleformat{\section}[hang]{\sihao\hei\sf}{\sihao\hei\thesection}{1em}{}{}
\titleformat{\section}[hang]{\song\xiaosi\hei\sf}{\xiaosi\hei\thesection}{1em}{}{}
\titlespacing{\section}{0pt}{0.5ex}{0.95ex}
% \titleformat{\subsection}[hang]{\xiaosi\hei\sf}{\xiaosi\hei\thesubsection}{1em}{}{}
\titleformat{\subsection}[hang]{\song\xiaosi\hei\sf}{\xiaosi\hei\thesubsection}{1em}{}{}
\titlespacing{\subsection}{0pt}{0.5ex}{0.95ex}


% 缩小目录中各级标题之间的缩进
\contentsmargin{0pt}
\dottedcontents{chapter}[0cm]{}{1.0em}{3pt}
\dottedcontents{section}[1cm]{}{1.8em}{3pt}
\dottedcontents{subsection}[2cm]{}{2.7em}{3pt}

% 段落之间的竖直距离
\setlength{\parskip}{0ex}
% 段落缩进
\setlength{\parindent}{2em}
% 定义行距
\renewcommand{\baselinestretch}{1.25}
% 参考文献条目间行间距
\setlength{\bibsep}{2pt}

%%%%%%%%%%%%%%%%%%%%%%%%%%%%%%%%%%%%%%%%%%%%%%%%%%%%%%%%%%%%%%%%%%%%%% 
% 页眉页脚设置
%%%%%%%%%%%%%%%%%%%%%%%%%%%%%%%%%%%%%%%%%%%%%%%%%%%%%%%%%%%%%%%%%%%%%% 

\newcommand{\makeheadrule}{%
  \makebox[0pt][l]{\rule[.7\baselineskip]{\headwidth}{0.5pt}}%
  \vskip-.8\baselineskip}

\makeatletter
\renewcommand{\headrule}{%
  {\if@fancyplain\let\headrulewidth\plainheadrulewidth\fi
      \makeheadrule}}

\pagestyle{fancyplain}

\fancyhf{}
\fancyhead[CO]{\song\wuhao\@ctitle}
\fancyhead[CE]{\song\wuhao\@ctitle}
\fancyfoot[C,C]{\xiaowu$-$~\thepage~$-$}

% Clear Header Style on the Last Empty Odd pages
\makeatletter
\def\cleardoublepage{\clearpage\if@twoside \ifodd\c@page\else%
      \hbox{}%
      \thispagestyle{empty}%              % Empty header styles
      \newpage%
      \if@twocolumn\hbox{}\newpage\fi\fi\fi}



%%%%%%%%%%%%%%%%%%%%%%%%%%%%%%%%%%%%%%%%%%%%%%%%%%%%%%%%%%%%%%%%%%%%%% 
% 列表环境设置

%%%%%%%%%%%%%%%%%%%%%%%%%%%%%%%%%%%%%%%%%%%%%%%%%%%%%%%%%%%%%%%%%%%%%% 

\setlist[enumerate]{(1),itemsep=-5pt,topsep=0mm,labelindent=\parindent,leftmargin=*}


%%%%%%%%%%%%%%%%%%%%%%%%%%%%%%%%%%%%%%%%%%%%%%%%%%%%%%%%%%%%%%%%%%%%%% 
% 国际单位,以点连接。
%%%%%%%%%%%%%%%%%%%%%%%%%%%%%%%%%%%%%%%%%%%%%%%%%%%%%%%%%%%%%%%%%%%%%% 
\sisetup{inter-unit-product = { }\cdot{ }}

%%%%%%%%%%%%%%%%%%%%%%%%%%%%%%%%%%%%%%%%%%%%%%%%%%%%%%%%%%%%%%%%%%%%%% 
% 参考文献的处理
%%%%%%%%%%%%%%%%%%%%%%%%%%%%%%%%%%%%%%%%%%%%%%%%%%%%%%%%%%%%%%%%%%%%%% 

% \addtolength{\bibsep}{-0.5em}              % 缩小参考文献间的垂直间距
\setlength{\bibhang}{2em}
\bibpunct{[}{]}{,}{s}{}{}




% \let\orig@Itemize =\itemize
% \let\orig@Enumerate =\enumerate
% \let\orig@Description =\description

% \def\Myspacing{\itemsep=1ex \topsep=-4ex \partopsep=-2ex \parskip=-1ex \parsep=2ex}
% \def\newitemsep{
% \renewenvironment{itemize}{\orig@Itemize\Myspacing}{\endlist}
% \renewenvironment{enumerate}{\orig@Enumerate\Myspacing}{\endlist}
% \renewenvironment{description}{\orig@Description\Myspacing}{\endlist}
% }
%   \def\olditemsep{
%   \renewenvironment{itemize}{\orig@Itemize}{\endlist}
%   \renewenvironment{enumerate}{\orig@Enumerate}{\endlist}
%   \renewenvironment{description}{\orig@Description}{\endlist}
% }
%   \renewcommand{\labelenumi}{(\arabic{enumi})}
%   \newitemsep

%%%%%%%%%%%%%%%%%%%%%%%%%%%%%%%%%%%%%%%%%%%%%%%%%%%%%%%%%%%%%%%%%%%%%%   
%   其他设置
%%%%%%%%%%%%%%%%%%%%%%%%%%%%%%%%%%%%%%%%%%%%%%%%%%%%%%%%%%%%%%%%%%%%%%   
%   增加 \ucite 命令使显示的引用为上标形式
%   \newcommand{\ucite}[1]{$^{\mbox{\scriptsize \cite{#1}}}$}

%%%%%%%%%%%%%%%%%%%%%%%%%%%%%%%%%%%%%%%%%%%%%%%%%%%%%%%%%%%%%%%%%%%%%%   
%   图形表格
%%%%%%%%%%%%%%%%%%%%%%%%%%%%%%%%%%%%%%%%%%%%%%%%%%%%%%%%%%%%%%%%%%%%%%   
\renewcommand{\figurename}{图}
\renewcommand{\tablename}{表}
\DeclareCaptionFont{song}{\song\wuhao}
\captionsetup{font=song,labelsep=quad}


\newcommand{\tablepage}[2]{\begin{minipage}{#1}\vspace{0.5ex} #2 \vspace{0.5ex}\end{minipage}}
\newcommand{\returnpage}[2]{\begin{minipage}{#1}\vspace{0.5ex} #2 \vspace{-1.5ex}\end{minipage}}


%%%%%%%%%%%%%%%%%%%%%%%%%%%%%%%%%%%%%%%%%%%%%%%%%%%%%%%%%%%%%%%%%%%%%% 
% 定义题头格言的格式
%%%%%%%%%%%%%%%%%%%%%%%%%%%%%%%%%%%%%%%%%%%%%%%%%%%%%%%%%%%%%%%%%%%%%% 

\newsavebox{\AphorismAuthor}
\newenvironment{Aphorism}[1]
{\vspace{0.5cm}\begin{sloppypar} \slshape
    \sbox{\AphorismAuthor}{#1}
    \begin{quote}\small\itshape }
      {\\ \hspace*{\fill}------\hspace{0.2cm} \usebox{\AphorismAuthor}
    \end{quote}
  \end{sloppypar}\vspace{0.5cm}}

% 自定义一个空命令,用于注释掉文本中不需要的部分。
\newcommand{\comment}[1]{}

% This is the flag for longer version
\newcommand{\longer}[2]{#1}

\newcommand{\ds}{\displaystyle}

% define graph scale
\def\gs{1.0}

%%%%%%%%%%%%%%%%%%%%%%%%%%%%%%%%%%%%%%%%%%%%%%%%%%%%%%%%%%%%%%%%%%%%%%%%%%%%%%%% 
% 封面摘要
%%%%%%%%%%%%%%%%%%%%%%%%%%%%%%%%%%%%%%%%%%%%%%%%%%%%%%%%%%%%%%%%%%%%%%%%%%%%%%%% 
\def\cdegree#1{\def\@cdegree{#1}}\def\@cdegree{}
\def\ctitle#1{\def\@ctitle{#1}}\def\@ctitle{}
\def\caffil#1{\def\@caffil{#1}}\def\@caffil{}
\def\csubject#1{\def\@csubject{#1}}\def\@csubject{}
\def\cauthor#1{\def\@cauthor{#1}}\def\@cauthor{}
\def\cauthorno#1{\def\@cauthorno{#1}}\def\@cauthorno{}
\def\csupervisor#1{\def\@csupervisor{#1}}\def\@csupervisor{}
\def\cdate#1{\def\@cdate{#1}}\def\@cdate{}
\long\def\cabstract#1{\long\def\@cabstract{#1}}\long\def\@cabstract{}
\def\ckeywords#1{\def\@ckeywords{#1}}\def\@ckeywords{}
\def\etitle#1{\def\@etitle{#1}}\def\@etitle{}
\long\def\eabstract#1{\long\def\@eabstract{#1}}\long\def\@eabstract{}
\def\ekeywords#1{\def\@ekeywords{#1}}\def\@ekeywords{}
\def\department#1{\def\@department{#1}}\def\@department{}
\def\teacher#1{\def\@teacher{#1}}\def\@teacher{}
% 封面
\def\makecover{
  \begin{titlepage}
    \newpage
    \thispagestyle{empty}
    %在这里有一个日语空格
    \begin{center}
      \parbox[t][2.80cm][c]{\textwidth}
      {
        \vspace{1.3cm}
        \linespread{1.25}
        \begin{center}
          {\xiaoer\song\textbf{\@cdegree}\\}
          \vspace{35pt}
          {\erhao\song\textbf{\@ctitle}\\}
          {\sanhao\textbf{\@etitle}\\}
        \end{center}
      }
      \parbox[b][15cm][c]{\textwidth}
      {
        \vspace{14.4cm}
        \begin{center}
          {
            \xiaosi\song
            \renewcommand\arraystretch{1.2}
            \begin{tabular}
              {p{0cm}p{70pt}   @{\extracolsep{0.5em}}lc}

              ~ & 学 \hfill 部 \hfill(院):        & \fixedunderlineleft[5.07cm] {\@department} & \\
              ~ & 专 \hfill 业:                     & \fixedunderlineleft[5.07cm]{\@csubject}    & \\
              ~ & 学 \hfill 生 \hfill 姓 \hfill 名: & \fixedunderlineleft[5.07cm]{\@cauthor}     & \\
              ~ & 学 \hfill 号:                     & \fixedunderlineleft[5.07cm]{\@cauthorno}   & \\
              ~ & 指 \hfill 导 \hfill 教 \hfill 师: & \fixedunderlineleft[5.07cm]{\@csupervisor} & \\
              ~ & 完 \hfill 成 \hfill 日 \hfill 期: & \fixedunderlineleft[5.07cm]{\@cdate}       & \\
            \end{tabular}
          }
        \end{center}
      }
      \parbox[b][3cm][c]{\textwidth}
      {
        \centering
        \vspace{3.7cm}
        {
          \song\xiaosi
          大连理工大学\\
        }
      }
    \end{center}
    %    \cleardoublepage
  \end{titlepage}
}



\def\makeabstract{
\defaultfont
\addcontentsline{toc}{chapter}{摘  要}
\setcounter{page}{1}
\defaultfont
\linespread{1.25}
\begin{center}
  {\song\sanhao\textbf {\@ctitle }}\\
  {\xiaosan Lila Setiyani, Abin Syamsudin, Abdorrakhman Gintings, Daeng Arifin}\\
  {\song\sihao 印度尼西亚STMIIK ROSMA信息系统部,印度尼西亚努桑塔拉伊斯兰大学教育管理系}
\end{center}

\@cabstract
\vspace{0.53cm}

\noindent {\hei\textbf{{关键词:{\@ckeywords}}}}
}

\makeatletter
\def\hlinewd#1{%
\noalign{\ifnum0=`}\fi\hrule \@height #1 \futurelet
  \reserved@a\@xhline}
\makeatother

% 定义索引生成
\def\generateindex
{
  \addcontentsline{toc}{chapter}{\indexname}
  \printindex
  \cleardoublepage
}

\raggedbottom