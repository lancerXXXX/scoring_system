% !TEX TS-program = XeLaTeX
% !TEX encoding = UTF-8 Unicode

%%%%%%%%%%%%%%%%%%%%%%%%%%%%%%%%%%%%%%%%%%%%%%%%%%%%%%%%%%%%%%%%%%%%%% 
% 
%	大连理工大学博士论文 XeLaTeX 模版 —— 格式文件 format.tex
%	版本:0.1beta1
%	最后更新:2018.5.23
%	修改者:王天宇 (E-mail: heavenlykingwty@gmail.com)
%	编译环境:macOS MacTex XeLatex (Texpad)
% 
%%%%%%%%%%%%%%%%%%%%%%%%%%%%%%%%%%%%%%%%%%%%%%%%%%%%%%%%%%%%%%%%%%%%%% 


\cdegree{\makebox[11.2cm][s]{大连理工大学本科外文翻译}}
\ctitle{大学生论文信息系统管理功能需求分析:印尼高校的案例分析}
\etitle{The Analysis of Functional Needs on Undergraduate Thesis Information System Management: A Case Study in Indonesian Universities}

% 根据需要添加字符间距
\department{\makebox[5.07cm][c]{软件学院}}
\csubject{\makebox[5.07cm][c]{软件工程}}
\cauthor{\makebox[5.07cm][c]{刘树东}}
\cauthorno{\makebox[5.07cm][c]{201792409}}
\csupervisor{\makebox[5.07cm][c]{李明楚}}
\cdate{\makebox[5.07cm][c]{\today}}

% 这里默认使用最后编译的时间,也可自行给定日期,注意汉字和数字之间的空格。
%\cdate{{\quad\;}\the\year~年~\the\month~月~\the\day~日}

\cabstract{
    \noindent 摘要:本科毕业论文是学生获得学士学位必须完成的科研论文。在完成毕业论文的过程中,很多学生在确定研究课题、督导、管理流程等方面遇到了困难,以至于不能按时毕业。本研究旨在分析学生和管理人员都需要的本科生毕业论文信息系统管理的功能需求。该系统的作用是加快毕业论文的准备过程,对学生、督导和论文管理管理人员都有帮助。本研究的程序是通过编制矩阵规划和数据收集来进行的。作为数据来源,本研究涉及20名正在撰写本科毕业论文的学生和10名已经完成毕业论文的校友。本样本的选取标准为参与过本科毕业论文业务流程的学生或校友。在抽取样本的过程中,研究者采用了目的性抽样的方法。此外,导师和研究计划管理人员也对论文管理中的困难进行了探讨。此外,通过PIECES(Performance,Information,Economy,Control,Efficiency,Service)对数据进行分析。本研究得出了一份关于毕业论文管理系统的功能需求清单,以鼓励学生按时毕业,其中高校可以实施该系统,并向用户进行社会化推广。

}

\ckeywords{本科论文分析,功能需求信息,系统管理}

\makecover