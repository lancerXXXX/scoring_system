% !TEX TS-program = XeLaTeX
% !TEX encoding = UTF-8 Unicode

\chapter{介绍}
\label{介绍}
\defaultfont

印度尼西亚高等教育机构的质量由国家高等教育认证委员会(BAN-PT)进行的认证来衡量,并以认证等级的形式呈现。该等级是根据既定的标准来衡量的,包括学生和毕业生的部分。这些部分是根据学生按时毕业的比例来衡量的[1]。Widarto[2]在印度尼西亚一所大学进行的研究显示,学生在按时完成学业方面受到一些挑战的制约,包括从家到校园的距离较远,难以接受监督,学生仍有课程的学习负担,一些学生还在工作。此外,学生也提出,毕业论文指导教师的选择往往与学生的研究课题不一致。因此,高校要想鼓励学生按时完成学业,有效、高效的论文管理是必不可少的。Carl Marnewick[3]认为,使用信息管理系统可以在管理业务流程方面提供优势,并以可持续的方式增强组织的优势。组织中实施的信息系统可以改变业务流程,使其更加有效和高效,也可以鼓励组织有更好的管理[4]。
本研究旨在分析有效、高效的本科生毕业论文信息管理系统的功能需求,以鼓励学生按时毕业。毕业论文信息系统管理的功能需求结果,有利于高职院校提高学生按时毕业的比例,提高院校的认证水平。下面就构成本研究的一些理论进行讨论。本研究对于了解未毕业论文业务流程中各参与方所面临的障碍或问题具有非常重要的意义。通过对问题的界定,将为能够提高学生按时毕业的比例提供机会,从而对完成高等教育认证产生影响。

\section{软件工程}

软件工程是一种社会活动,它涉及到大量的互动,如软件开发团队的成员以不同的方式和任务相互协作,如系统分析师、程序员、利益相关者等。[5]. 软件工程被描述为一种方法,它规范了技术和管理方面的努力,将利益相关者的一系列需求、期望和约束转化为解决方案[6]。在这个工程过程中,应用了软件开发生命周期(SDLC)、敏捷等几种软件开发过程模型来对软件开发人员进行监督。在每个过程模型中,信息系统需求的定义阶段是非常重要的。在这种情况下,定义过程被认为是确定利益相关者问题具体解决方案的关键。根据Melegati所引用的Nuseibeh和Easterbrook的观点,将需求定义为:通过识别利益相关者及其需求,发现一个软件的目的,并将发现记录下来,以便将来进行分析、交流和实施的过程[14][15]。

\section{需求分类}

软件开发过程模型中的信息系统需求来源[6],可分为两种。第一,功能需求,描述要开发或构建的系统的功能。这个系统的功能是希望能成为利益相关者所期望的解决方案。第二,非功能需求是指在捕捉系统运行所使用的属性方面所期望的产品规格。

\section{获取利益相关者需求的实践}

有研究者[7]、[9]概述,用于获取利益相关者需求的做法包括访谈、问卷调查、观察等。这些技术或方法各有优缺点。方法的选择可以根据利益相关者的条件进行调整。
Kandaga \& Felix(2011)在研究中成功开发了基于网络的最终项目管理应用,但没有彻底讨论确定系统功能需求的过程[12]。Likewise Simatupang \& Muhammad也开发了一个基于移动的最终项目管理应用,但也没有关注详细的系统需求分析过程。本研究是对系统开发的基本过程进行详细说明的研究,因为系统或应用的开发如果不从与系统需求相关的深入分析开始,将无法为用户带来好处[13]。
