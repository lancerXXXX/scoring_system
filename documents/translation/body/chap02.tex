% !TEX TS-program = XeLaTeX
% !TEX encoding = UTF-8 Unicode

\chapter{基于多直方图的人脸识别}
\label{chap02}
\defaultfont
该部分将提出一种基于多直方图的人脸识别方法。这里的多重直方图都是来源于图像的某一行,但是这个概念是在一个通用的公式中提出的,允许多重直方图来自图像的不同部分。在后继中,我们提出了图像直方图的概念和图像的多个部分的直方图,特别是图像的行。然后,我们提出了多重直方图之间的距离度量。
\section{图像直方图的理论定义}
假设我们有一个包含N行M列的图像I(x,y),像素总数为N×M,假设每个像素点I(x,y)用8位表示,图像是单色图像,则I(x,y)=r∈[0…255]。然后,对于索引为i∈[1,NM]且坐标为(x,y)的像素点,颜色强度值表示为I(i)=I(x,y)=r。

图像直方图是表示离散变量r所有可能值出现次数的向量H(r)。给定图像l(x,y)的直方图HI(r),可以使用如下向量来表示:
\begin{equation}
fomular
\end{equation}
向量中的每一项元素为:
\begin{equation}
fomular
\end{equation}
这项工作的一种主要的特点,是从人脸图像中使用单个行的直方图s∈[1,N]进行识别。某行的直方图可以写为
\begin{equation}
fomular
\end{equation}
向量中的每一项元素为:
\begin{equation}
fomular
\end{equation}
\section{基于多重直方图距离的人脸识别}
本文提出了一种新的基于标准差的人脸图像距离度量方法。它具有高度的区分性,易于在硬件平台上实现。第4节中的经验证明,所提出的方法受光照条件的变化影响较小。图像I中,第s行的直方图HI(s)的标准差$STD_I$给出了行s中整个像素的强度出现与平均出现值的接近程度的概念。具有较大标准差的直方图分布在广泛的范围内。图像I中第s行的直方图HI(s)的标准差$STD_I$在数学上定义如下:
\begin{equation}
fomular
\end{equation}
其中(Hi) ̅是行直方图HI(r,s)的平均值。最后,计算一个向量$STD_I$,它包含N个分量,每个分量代表图像I中某行的$STD_I$。

给定一个查询人脸图像Q,任务是计算该图像到数据库中每个图像D的距离指数。我们提出的距离度量技术是$STD_I$和$STD_I$之间差分向量的$STD_I$范数。这个距离也被称为曼哈顿距离度量。这些分别是查询图像和来自数据库的图像的标准差向量。当在线计算$STD_I$时,离线计算对应于数据库中每个图像的矢量$STD_I$。

数学上,查询人脸“Q”和数据库人脸“D”之间的距离指数(DS)定义如下:
\begin{equation}
fomular
\end{equation}
其中N是两幅图像中每幅图像的行数,$STD_I$是查询人脸图像第s行直方图的标准差,$STD_I$是数据库人脸图像第s行直方图的标准差。

下一节将概述所提出的算法。此外,我们将在下一节中展示,所提出的距离度量受光照条件改变影响较小,对估计图像直方图的标准差没有显著影响。众所周知的伽马校正方法被用来给图像增加渲染效果.
\section{人脸识别技术}
如在第一节中讨论的,我们的面部识别技术涉及一个到多个匹配,将一个查询人脸和在录入数据库中的多个人脸作比较。在每一次比较中,都会计算一个距离指数(“在本节中描述后”)。最低指数是指数据库中与查询最相似的人脸。我们的技术假设如下:
\begin{figure}
	\centering
	\caption{\song\wuhao 人脸识别算法框图}

\end{figure}
(1)使用者愿意合作,但他无法在相机下呈现面部。换言之,我们的技术不考虑人脸面部姿势、表情、年龄跨度、头发、装扮和移动。

(2)由于环境光照是大多数人脸识别应用最主要的挑战,我们的技术在计算距离指数时考虑了这一因素。因此,改变亮度不会影响该技术的性能。用于改变图像的环境光照的方法是伽马校正。

(3)所有的人脸都用灰度表示,即每个像素表示为8位值(0到255)

如图1所示,我们提出的人脸识别算法经过以下步骤:
第一步,将新的查询人脸与注册数据库中的多个人脸比较,根据公式(4)和(2)分别计算查询面的每一行的直方图。为此,每一行被视为一个像素数组,大小等价于图像列的数量。每个像素可以是0到255之间的任意数字(像素值)。行数组从第一个像素开始扫描到最后一个像素,使用256个计数器。每个像素值独占一个计数器。当每个计数器遇到相同值的新像素时,该计数器数值加一。扫描结束时,每个计数器都存储了某一像素值(bin count)出现的次数,并存储在数组“hist”中。直方图计算过程的硬件实现见第5节。此外,算法1给出了用于计算图像行的直方图的伪代码。在计算所有行的直方图之后,根据公式(5)计算每行直方图的标准差。

对于任何新的人脸、对象等的数据库,采用离线(在设计时)计算直方图和标准差。因此,花多长时间并不重要.

在下一步中,查询人脸和数据库人脸之间的距离指数将按照公式(6)所示方法进行计算。它反映了查询人脸与数据库人脸的接近程度。指数越低,则匹配程度越高。对数据库中注册的每个人脸图像重复此步骤,并计算其与查询人脸的距离指数。在我们的技术的最后一步中,比较器将比较距离指数并找到最小值。最小指数就是更接近查询人脸的数据库人脸。完整的硬件实现见第5节。

