% !TEX TS-program = XeLaTeX
% !TEX encoding = UTF-8 Unicode

\chapter{介绍}
\label{介绍}
\defaultfont

印度尼西亚高等教育机构的质量由国家高等教育认证委员会(BAN-PT)进行的认证来衡量,并以认证等级的形式呈现。该等级是根据既定的标准来衡量的,包括学生和毕业生的部分。这些部分是根据学生按时毕业的比例来衡量的[1]。Widarto[2]在印度尼西亚一所大学进行的研究显示,学生在按时完成学业方面受到一些挑战的制约,包括从家到校园的距离较远,难以接受监督,学生仍有课程的学习负担,一些学生还在工作。此外,学生也提出,毕业论文指导教师的选择往往与学生的研究课题不一致。因此,高校要想鼓励学生按时完成学业,有效、高效的论文管理是必不可少的。Carl Marnewick[3]认为,使用信息管理系统可以在管理业务流程方面提供优势,并以可持续的方式增强组织的优势。组织中实施的信息系统可以改变业务流程,使其更加有效和高效,也可以鼓励组织有更好的管理[4]。
本研究旨在分析有效、高效的本科生毕业论文信息管理系统的功能需求,以鼓励学生按时毕业。毕业论文信息系统管理的功能需求结果,有利于高职院校提高学生按时毕业的比例,提高院校的认证水平。下面就构成本研究的一些理论进行讨论。本研究对于了解未毕业论文业务流程中各参与方所面临的障碍或问题具有非常重要的意义。通过对问题的界定,将为能够提高学生按时毕业的比例提供机会,从而对完成高等教育认证产生影响。

\section{软件工程}

软件工程是一种社会活动,它涉及到大量的互动,如软件开发团队的成员以不同的方式和任务相互协作,如系统分析师、程序员、利益相关者等。[5]. 软件工程被描述为一种方法,它规范了技术和管理方面的努力,将利益相关者的一系列需求、期望和约束转化为解决方案[6]。在这个工程过程中,应用了软件开发生命周期(SDLC)、敏捷等几种软件开发过程模型来对软件开发人员进行监督。在每个过程模型中,信息系统需求的定义阶段是非常重要的。在这种情况下,定义过程被认为是确定利益相关者问题具体解决方案的关键。根据Melegati所引用的Nuseibeh和Easterbrook的观点,将需求定义为:通过识别利益相关者及其需求,发现一个软件的目的,并将发现记录下来,以便将来进行分析、交流和实施的过程[14][15]。

\begin{equation}
	fomular
\end{equation}
向量中的每一项元素为:
\begin{equation}
	fomular
\end{equation}
这项工作的一种主要的特点,是从人脸图像中使用单个行的直方图s∈[1,N]进行识别。某行的直方图可以写为
\begin{equation}
	fomular
\end{equation}
向量中的每一项元素为:
\begin{equation}
	fomular
\end{equation}
\section{基于多重直方图距离的人脸识别}
本文提出了一种新的基于标准差的人脸图像距离度量方法。它具有高度的区分性,易于在硬件平台上实现。第4节中的经验证明,所提出的方法受光照条件的变化影响较小。图像I中,第s行的直方图HI(s)的标准差$STD_I$给出了行s中整个像素的强度出现与平均出现值的接近程度的概念。具有较大标准差的直方图分布在广泛的范围内。图像I中第s行的直方图HI(s)的标准差$STD_I$在数学上定义如下:
\begin{equation}
	fomular
\end{equation}
其中(Hi) ̅是行直方图HI(r,s)的平均值。最后,计算一个向量$STD_I$,它包含N个分量,每个分量代表图像I中某行的$STD_I$。

给定一个查询人脸图像Q,任务是计算该图像到数据库中每个图像D的距离指数。我们提出的距离度量技术是$STD_I$和$STD_I$之间差分向量的$STD_I$范数。这个距离也被称为曼哈顿距离度量。这些分别是查询图像和来自数据库的图像的标准差向量。当在线计算$STD_I$时,离线计算对应于数据库中每个图像的矢量$STD_I$。

数学上,查询人脸“Q”和数据库人脸“D”之间的距离指数(DS)定义如下:
\begin{equation}
	fomular
\end{equation}
其中N是两幅图像中每幅图像的行数,$STD_I$是查询人脸图像第s行直方图的标准差,$STD_I$是数据库人脸图像第s行直方图的标准差。

下一节将概述所提出的算法。此外,我们将在下一节中展示,所提出的距离度量受光照条件改变影响较小,对估计图像直方图的标准差没有显著影响。众所周知的伽马校正方法被用来给图像增加渲染效果.
\section{人脸识别技术}
如在第一节中讨论的,我们的面部识别技术涉及一个到多个匹配,将一个查询人脸和在录入数据库中的多个人脸作比较。在每一次比较中,都会计算一个距离指数(“在本节中描述后”)。最低指数是指数据库中与查询最相似的人脸。我们的技术假设如下:
\begin{figure}
	\centering
	\caption{\song\wuhao 人脸识别算法框图}

\end{figure}
(1)使用者愿意合作,但他无法在相机下呈现面部。换言之,我们的技术不考虑人脸面部姿势、表情、年龄跨度、头发、装扮和移动。

(2)由于环境光照是大多数人脸识别应用最主要的挑战,我们的技术在计算距离指数时考虑了这一因素。因此,改变亮度不会影响该技术的性能。用于改变图像的环境光照的方法是伽马校正。

(3)所有的人脸都用灰度表示,即每个像素表示为8位值(0到255)

如图1所示,我们提出的人脸识别算法经过以下步骤:
第一步,将新的查询人脸与注册数据库中的多个人脸比较,根据公式(4)和(2)分别计算查询面的每一行的直方图。为此,每一行被视为一个像素数组,大小等价于图像列的数量。每个像素可以是0到255之间的任意数字(像素值)。行数组从第一个像素开始扫描到最后一个像素,使用256个计数器。每个像素值独占一个计数器。当每个计数器遇到相同值的新像素时,该计数器数值加一。扫描结束时,每个计数器都存储了某一像素值(bin count)出现的次数,并存储在数组“hist”中。直方图计算过程的硬件实现见第5节。此外,算法1给出了用于计算图像行的直方图的伪代码。在计算所有行的直方图之后,根据公式(5)计算每行直方图的标准差。

对于任何新的人脸、对象等的数据库,采用离线(在设计时)计算直方图和标准差。因此,花多长时间并不重要.

在下一步中,查询人脸和数据库人脸之间的距离指数将按照公式(6)所示方法进行计算。它反映了查询人脸与数据库人脸的接近程度。指数越低,则匹配程度越高。对数据库中注册的每个人脸图像重复此步骤,并计算其与查询人脸的距离指数。在我们的技术的最后一步中,比较器将比较距离指数并找到最小值。最小指数就是更接近查询人脸的数据库人脸。完整的硬件实现见第5节。

