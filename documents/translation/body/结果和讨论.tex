% !TEX TS-program = XeLaTeX
% !TEX encoding = UTF-8 Unicode

\chapter{结果和讨论}
\label{结果和讨论}
\defaultfont


\section{规划}
\begin{table}[htbp]
      \centering
      \song\wuhao
      \caption{数据收集矩阵}
      \label{数据收集矩阵}
      \begin{tabular}{lll}
            \hline
            数据/信息                                     & 数据源                 & 数据收集技术        \\ \hline
            \multirow{3}{*}{本科论文管理规划中的业务流程} & 1.        高等教育管理 & 1.        面试      \\
                                                          & 2.        学生         & 2.        观察      \\
                                                          &                        & 3.        研究文件  \\ \hline
            \multirow{3}{*}{实施本科论文管理的业务流程}   & 1.       高等教育管理  & 1.        面试      \\
                                                          & 2.        学生         & 2.        观察      \\
                                                          &                        & 3.        研究文件  \\ \hline
            \multirow{3}{*}{本科论文评价管理中的业务流程} & 1.        高等教育管理 & 1.        面试      \\
                                                          & 2.        学生         & 2.        观察      \\
                                                          &                        & 3.        研究 文件 \\ \hline
      \end{tabular}
\end{table}

数据或信息的确定是在本科毕业论文管理的基础上提取的。在毕业论文管理中,会出现计划、实施、评价等三个过程。根据这些过程,对相关数据或信息进行整理。同时,根据这些过程中所涉及的用户,确定数据的来源。根据数据来源调整数据收集技术,这些数据来源包括访谈、观察和研究文献。数据或信息、数据来源和数据收集技术将成为制作数据收集工具的基础。根据汇总表中的数据或信息,在数据收集过程中向信息提供人员提出一些问题。以下是研究工具的安排:

\begin{table}[htbp]
      \centering
      \song\wuhao
      \caption{数据收集工具(访谈)}
      \label{数据收集工具访谈}
      \begin{tabular}{ll}
            \hline
            问题                             & 信息提供者的回答                           \\ \hline
            \multicolumn{2}{l}{论文管理规划}                                              \\ \hline
            论文管理规划流程如何进行?       & 在业务流程图\ref{数据收集矩阵}中解释       \\
            论文管理规划存在哪些制约因素?   & 在寻找研究课题思路的过程中遇到的困难。     \\
            论文管理规划流程涉及哪些文件?   & 论文报名表、论文提案评审表                 \\ \hline
            \multicolumn{2}{l}{论文管理实现}                                              \\ \hline
            论文管理实施过程如何?           & 在业务流程图\ref{数据收集矩阵} 中解释      \\
            论文管理实施中遇到的哪些障碍?   & 指导过程不灵活,研究负责人难以控制论文进度 \\
            论文管理过程中涉及哪些文件?     & 指导手册                                   \\ \hline
            \multicolumn{2}{l}{论文管理评价}                                              \\ \hline
            论文管理评价流程如何进行?       & 指导手册                                   \\
            论文管理评价中遇到的障碍是什么? & 所进行的评价没有注重论文成果的实用性       \\
            论文管理评价过程中涉及哪些文件? & 关键评估表                                 \\ \hline
      \end{tabular}
\end{table}

在数据收集工具和观察技术方面,虽然问题清单不同,但都采用了访谈。观察问题是根据研究者的直接观察结果来回答的。同时,研究文献工具涉及到每一个归档论文管理过程中的文件清单。

\section{数据收集结果}

访谈、观察和研究文献作为数据收集工具。在访谈过程中,研究者对正在进行毕业论文课程的学生和已经完成毕业论文的校友两类被调查者进行了访谈,包括管理层和学生。同时,为了保证访谈信息的有效性,研究者还进行了观察和学习记录。

\section{数据分析}

然后对数据收集活动的结果进行分析,确定本科毕业论文管理业务流程,确定问题,确定因果关系,确定问题解决方案,最后确定功能需求。

\section{业务流程的识别结果}

通过对毕业论文管理中的流程进行定义分析。分析结果以流程图的形式进行建模。从数据收集的结果来看,可以确定本科毕业论文管理中发生的4个业务流程,包括:1)毕业论文方案管理;2)毕业论文指导管理;3)毕业论文研讨管理;4)毕业论文答辩管理。

\section{利用碎片框架的问题识别结果}

通过编制PIECES矩阵,利用PIECES框架进行问题识别,具体如下:

\begin{table}[htbp]
      \centering
      \song\wuhao
      \caption{PIECES分析矩阵}
      \label{PIECES分析矩阵}
      \begin{tabular}{lp{13cm}}
            \hline
            类别 & 问题识别                                                                                                   \\ \hline
            性能 & 检索论文题目和研究对象的过程需要很长的时间。提案、研讨、答辩的评审人和指导人不适合学生进行的研究领域。     \\
            信息 & 在每个数据存储过程中生成的信息(提案、监督、研讨、答辩)在每个过程中都是不准确的。                         \\
            经济 & 行政档案办理(提案、监督、研讨、答辩)仍采用传统方式。面对面的监督需要时间和金钱。行政活动需要时间和金钱。 \\
            控制 & 存储数据(提案、监督、研讨、答辩)缺乏安全检测。                                                           \\
            效率 & 学生在论文提案、指导、研讨、答辩过程中的数据处理需要时间。                                                 \\
            服务 & 为学生和管理者的需求提供信息(提案、监督、研讨会、答辩)的服务。                                           \\ \hline
      \end{tabular}
\end{table}

\section{因果鉴定结果}

在利用PIECES框架识别问题的基础上,进行因果识别,确定问题的原因。以下是对导致问题的因素的识别:

\begin{table}[htbp]
      \centering
      \song\wuhao
      \caption{导致问题的因素}
      \label{导致问题的因素}
      \begin{tabular}{p{6cm}p{6cm}}
            \hline
            \multicolumn{2}{l}{性能}                                                                                                            \\ \hline
            问题                                                                 & 原因                                                         \\
            搜索研究课题和对象的过程需要时间                                     & 没有集思广益的媒体。                                         \\
            评审员/考官和监督员在每个过程中的能力领域有时与学生讨论的主题无关。  & 评审员/审查员和监督员的数据不完整。                          \\ \hline
            \multicolumn{2}{l}{信息}                                                                                                            \\ \hline
            问题                                                                 & 原因                                                         \\
            各个数据存储过程中(提案、监督、研讨、答辩)各环节产生的信息不准确。 & 有些数据没有很好地记录。                                     \\ \hline
            \multicolumn{2}{l}{经济}                                                                                                            \\ \hline
            问题                                                                 & 原因                                                         \\
            行政归档工作(提案、督办、研讨、答辩)仍采用成本高昂的常规方法。     & 管理归档过程仍使用纸张                                       \\
            面对面的监督需要时间和金钱                                           & 不能在线监督                                                 \\
            行政活动需要时间和金钱                                               & 行政程序在时间和地点方面不灵活                               \\ \hline
            \multicolumn{2}{l}{控制}                                                                                                            \\ \hline
            问题                                                                 & 原因                                                         \\
            存储的数据(建议、监督、研讨会和防御)缺乏安全测试                   & 存储数据的过程仍然使用工作表                                 \\ \hline
            \multicolumn{2}{l}{效率}                                                                                                            \\ \hline
            问题                                                                 & 原因                                                         \\
            学生在论文提案、指导、研讨、答辩过程中的数据处理需要时间。           & 数据输入过程仍使用工作表。当有人需要报告,他们应该撰写报告。 \\ \hline
            \multicolumn{2}{l}{服务}                                                                                                            \\ \hline
            问题                                                                 & 原因                                                         \\
            为学生和管理者的需求提供信息(提案、监督、研讨会、答辩)的服务。     & 手工管理
      \end{tabular}
\end{table}

为了更好地了解问题与原因之间的关系,根据表4中问题及其原因的确定,形成鱼骨图。以下是鱼骨图的分析结果:

\begin{figure}[htbp]
      \centering
      \label{}
      \caption{\song\wuhao 因果分析}
\end{figure}

\section{问题解决方案的识别结果}

\begin{table}[htbp]
      \centering
      \song\wuhao
      \caption{确定问题的解决方案}
      \label{确定问题的解决方案}
      \begin{tabular}{lp{9cm}}
      \end{tabular}
\end{table}

\section{功能需求的识别结果}

功能需求的确定是根据用户的需求来确定的,从解决问题的识别结果可以看出。根据本科生毕业论文管理过程中发生的业务流程的识别结果,有管理层、学生、讲师等三个用户。此外,下一步是通过绑定功能需求及其目标,确定本科生毕业论文管理的信息系统管理,如下表所示:

\begin{table}[htbp]
      \centering
      \song\wuhao
      \caption{大学生需求分析}
      \label{大学生需求分析}
      \begin{tabular}{lp{5cm}p{8cm}}
            \hline
              & 主要需求                         & 目标                                              \\ \hline
            1 & 焦点小组讨论 (FGD)   信息系统  & 加快搜索研究课题和对象的进程/加快登记提案的进程。 \\
            2 & 提案、研讨会和答辩管理信息系统。 & 简化报名程序,获取考试成绩和时间安排等信息。      \\
            3 & 在线监督信息系统                 & 简化监督程序                                      \\ \hline
      \end{tabular}
\end{table}

\begin{table}[htbp]
      \centering
      \song\wuhao
      \caption{讲师需求分析}
      \label{讲师需求分析}
      \begin{tabular}{lll}
            \hline
              & 主要需求                      & 目标                         \\ \hline
            1 & 讲师数据存储库信息系统        & 记录讲师的数据和专业知识规范 \\
            2 & 焦点小组讨论 (FGD) 信息系统 & 与学生交流思想               \\
            3 & 在线监督信息系统              & 简化监管                     \\ \hline
      \end{tabular}
\end{table}

\begin{table}[htbp]
      \centering
      \song\wuhao
      \caption{管理层需求分析}
      \label{管理层需求分析}
      \begin{tabular}{lll}
            \hline
              & 主要需求                         & 目标                                           \\ \hline
            1 & 讲师和学生数据资源库信息系统     & 显示讲师和学生数据                             \\
            2 & 行政信息系统(提案、研讨会、答辩) & 用于行政管理(登记、安排、提供与结果有关的信息) \\
            3 & 在线监督系统                     & 查看学生论文的进展情况                         \\ \hline
      \end{tabular}
\end{table}

\section{结论}

根据数据分析的结果,本科生毕业论文信息系统管理的功能需求有FGD媒体、在线管理、在线督导、主数据库(讲师和学生数据)等功能。本科生毕业论文信息系统管理的功能需求是基于深入访谈和观察的结果,与加密的业务流程相关,学生面临的制约因素如研究思路难以获得、指导过程不灵活、论文结果评价不准确,以及生产者难以进行控制学生论文进度等。这种功能需求分析可以作为高校在论文管理方面的参考。尽管目前还无法衡量该系统的实施效果,但该分析仍可为高校提供一个提高学生按时毕业比例的机会,从而提高院校的质量。



致谢:致谢已略(见原文)

参考文献:参考文献已略(见原文)