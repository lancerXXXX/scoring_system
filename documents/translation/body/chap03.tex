% !TEX TS-program = XeLaTeX
% !TEX encoding = UTF-8 Unicode

\chapter{稳定性与gamma方法}
\label{chap03}
\defaultfont
\section{伽马强度转换}
Gamma是控制图像强度的重要因素之一。因此,伽马校正是一种非线性特性,广泛用于调整亮度通道并生成具有不同亮度级别的图像[10]。伽马校正技术可视为直方图修正技术,可通过以下方法获得:
\begin{equation}
fomular
\end{equation}
其中μ是gamma值,x代表输入灰度值,f(x)是图像的输出gamma校正灰度值。这种方法比明亮的像素更能增强暗像素。图2显示了当图像的输入值从0变为1时,对应不同的μ值:0.5、1、2和3,输出的是伽马校正灰度值。

从这个图中,我们可以看到当μ=1时,输出值没有变化。要获得较暗的图像,μ应小于1。在这种情况下,暗像素值比亮像素值减少(变暗)更多(见图2,μ=0.5)。要获得更亮的图像,μ应大于1。在这种情况下,暗像素值比亮像素值增加(变得更亮)(参见图2,μ=2和μ=3)。
\section{环境光照改变对该方法的影响评价}
为了评估该人脸识别技术,我们使用不同μ{1,0.5,2,3}值的Gamma校正方法来改变查询图像的光线条件。这些值分别对应于“未改变光照条件”、“采用灰暗的光线条件”、“添加了明亮的光照条件”和“添加了更明亮的光照条件”。

如果查询人脸具有与数据库人脸不同的亮度级别,则直方图差异将导致错误的比较结果。这是因为图像的像素值基于μ值增加或减少。例如,图3所示不同μ值的查询人脸的第一行的直方图:在图3a中,未改变查询人脸的光照条件(μ=1),图中高度重复像素(bin)之一为“128”(bin count=43)。当暗环境下,改变μ=0.5(图3b),直方图向左移动(朝低像素值方向移动),高度重复的像素由“128”(bin count=43)变为“64”(具有相同的bin count)。将明亮的环境光照增加μ=2(图3c),直方图向右移动(朝向高像素值)。现在,高重复像素变为“181”,其频率增加(bin count=44)。将μ值增加到3(图3d),直方图移向高像素值。因此,高重复像素变为“202”,其频率增加(bin count=49)。

图3还揭示了不同μ值的查询人脸直方图的标准差:在图3a中,第一行直方图的标准差为25.92。在图3b中,STD几乎与在图3a中相同(实际上,两个值在小数部分的第三个数字不同),这是因为μ的变化非常小(从1到0.5)。当μ=2(图3c)时,STD略微增加到25.951,当μ=3(图3d)时,STD略微增加到26.285。

根据上述现象,我们得出结论,使用Gamma校正方法改变图像的亮度将改变直方图,但对图像的标准差影响较小,因此,如果查询人脸具有不同的亮度级别,则标准差是比较查询人脸和数据库人脸的一个更好的选择。




