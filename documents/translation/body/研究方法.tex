% !TEX TS-program = XeLaTeX
% !TEX encoding = UTF-8 Unicode

\chapter{研究方法}
\label{研究方法}
\defaultfont

本研究分为三个阶段,包括计划、数据收集和数据分析。在选择样本时,研究者考虑到样本的特点,即参与本科生毕业论文业务流程的行为者,在这种情况下,仅限于学生和校友,因为它是根据研究目标调整的。涉及的样本数量为20名正在进行毕业论文的学生和10名已经完成毕业论文的校友。本样本的选择是希望能够深入了解大学生毕业论文管理中的业务流程和遇到的障碍。以下是各阶段活动的详细内容。

\section{规划}

这项研究的程序从规划开始。进行这种规划是为了以正确的方式获得数据。在规划阶段,研究人员通过汇编以下文件为收集数据做准备:

\begin{itemize}
    \item 数据收集矩阵

          数据收集矩阵包含要提取的数据或信息、信息源、数据收集技术。
          数据采集汇总表包含待提取的数据或信息、信息来源、数据采集技术。根据毕业论文管理中的业务流程,包括注册、监理、评价等流程,可以确定需要提取的数据或信息。管理各业务流程的用户(利益相关者)成为信息来源。同时,将数据提取技术作为数据收集程序。数据收集过程中使用的技术包括访谈,作为一种沟通技术,用于探索确定利益相关者需求所需的数据或信息。访谈技术在本研究中采用的是开放性的访谈(Open intreview),目的是听取与本科生毕业论文管理业务流程相关的意见。其次,这种观察是为了验证在访谈阶段所探讨的业务流程是否遵循了信息提供者提供的信息而进行的技术。在这个业务流程中,观察到的事情是提案的编制、登记、指导和审查。除了编制汇总表外,还安排了一个活动时间表,以便按时完成项目,指导研究的每个阶段的实施。

    \item 数据收集工具

          数据收集工具是根据预先确定的数据收集矩阵安排的。该工具包含一个问题清单,重点是每个业务流程所采取的步骤、面临的制约因素和相关文件。预计通过了解这三个焦点,可以确定当前系统的流程、制约因素和流程图。

\end{itemize}

\section{数据收集}

数据收集以以前准备的工具为基础。研究对象是印度尼西亚的一所高等院校,即STMIK ROSMA Karawang。

\section{信息系统分析需要——数据分析}

这一阶段分为几个活动,以集中体现本分析的目的。以下是为确定毕业论文管理信息系统的功能需求规格而进行的活动。

\begin{figure}[htbp]
    \centering
    \label{业务流程图}
    \caption{\song\wuhao 业务流程(流程图)}
\end{figure}

\begin{itemize}
    \item 使用 PIECES 分析识别问题

          通过PIECES分析来识别问题,是从六个维度对前面活动中已经建模的现有业务流程进行评估。这些维度如下[10]:
          \begin{enumerate}
              \item 性能是用来衡量系统处理数据产生预期信息的可靠性的一个维度
              \item 信息是用来衡量信息或数据呈现的价值的一个维度。这个维度要考虑的重要组成部分是输出、输入和存储数据。
              \item 经济性是用于衡量投资价值和当前系统效益的维度。在这个维度中要考虑的组成部分是成本和利润。
              \item 控制是用于控制和保障当前系统的维度。
              \item 效率是用来衡量当前系统运行的效率水平的维度。可以用来衡量的参照物是操作系统和满足物料需求所需的用户数量。
              \item 服务是用来衡量当前系统的服务水平的一个维度。在评估这个维度时,需要考虑的事情有:准确度和一致性。
          \end{enumerate}
    \item 确定原因和影响

          因果识别过程以鱼骨图的形式进行建模。该图是由石川[11]开发的一种技术,用于识别、探索和描述具有特定原因的具体问题的原因。

    \item 确定问题的解决方案

          通过整理上一个活动中已经确定的问题的解决方案清单,来确定这个问题的解决方案。从因果分析活动中指定的每个问题中,将问题的根源分为输入、过程和输出等三类,进行解决方案的分析。

    \item 确定功能需求

          在确定功能需求时,需求的确定是基于来自利益相关者的用户。每项确定的需求都包括对该需求目的的解释。

\end{itemize}
