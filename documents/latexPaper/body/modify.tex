% !TEX TS-program = XeLaTeX
% !TEX encoding = UTF-8 Unicode

\chapter*{\hfill 修改记录 \hfill}
\addcontentsline{toc}{chapter}{修改记录}
\defaultfont
\linespread{1.25}
修改是论文写作过程中不可或缺的重要步骤,是提高论文质量的有效环节。修改的过程其实就是“去伪存真”、去糟粕取精华使论文不断“升华”的过程。

以下内容要求放到毕业设计(论文)修改记录中:

(1) 毕业设计(论文)题目修改

{\textbf {第一次修改记录:}}:(没有可删除,后面记录依次递进)

原题目:

修稿后题目:

(2) 指导教师变更

{\textbf {第二次修改记录:}}:(没有可删除,后面记录依次递进)

原指导教师:******更改后指导教师:******

(3) 校外毕业设计(论文)时间节点记录

{\textbf {第三次修改记录:}}:(没有可删除,后面记录依次递进)

本人于2019年1月申请到******大学做毕业设计(论文),指导教师为:******

校内指导教师为:******。2019年*月*日回到学校。

(4) 毕业设计(论文)内容重要修改记录

包括:指导教师要求的重大修改,评阅教师要求的修改,答辩委员会提出的修改意见以及检测后的修改记录等。

{\textbf {第四次修改记录:}}(如实记录重要修改,不可省略)

第5页2.1,{\textbf{修改前:}}

{\textbf{修改后:}}

{\textbf {第五次修改记录:}}

第8页表2.4表名,修改前:

{\textbf{修改后:}}

{\textbf {第六次修改记录:}}

(5) 毕业设计(论文)外文翻译修改记录

(6) 毕业设计(论文)正式检测重复比

修改记录正文选用模板中的样式所定义的“正文”,每段落首行缩进2字;字体:宋体,字号:小四,行距:多倍行距 1.25,间距:段前、段后均为0行。\\
\hspace*{8.5cm}记录人(签字):\\
\hspace*{8cm}指导教师(签字):