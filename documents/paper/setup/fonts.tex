% !TEX TS-program = XeLaTeX
% !TEX encoding = UTF-8 Unicode

%%%%%%%%%%%%%%%%%%%%%%%%%%%%%%%%%%%%%%%%%%%%%%%%%%%%%%%%%%%%%%%%%%%%%% 
% 
%	大连理工大学博士论文 XeLaTeX 模版 —— 格式文件 format.tex
%	版本:0.1beta1
%	最后更新:2018.5.23
%	修改者:王天宇 (E-mail: heavenlykingwty@gmail.com)
%	编译环境:macOS MacTex XeLatex (Texpad)
% 
%%%%%%%%%%%%%%%%%%%%%%%%%%%%%%%%%%%%%%%%%%%%%%%%%%%%%%%%%%%%%%%%%%%%%% 


% 英文字体设置特别推荐方案(Windows,需要安装 Adobe 字体),现代
\usepackage{fontspec}
\usepackage{xltxtra,xunicode}
\usepackage[CJKnumber,CJKchecksingle,BoldFont]{xeCJK}
\usepackage{amsmath}
\usepackage{amssymb}
\setmainfont[Mapping=tex-text]{Times New Roman}
\setsansfont[Mapping=tex-text]{Arial}
%\newfontfamily\ms{MS Sans Serif}
%\setmonofont{Consolas}


% 英文字体设置方案一(Windows,需要安装 LM10 字体),和 LaTeX 默认字体保持一致,经典
% \usepackage{amssymb}
% \usepackage{fontspec}
% \usepackage{amsmath}
% \usepackage[CJKnumber,CJKaddspaces,CJKchecksingle,BoldFont]{xeCJK}
% \usepackage{mathrsfs}   % 一种常用于定义泛函算子的花体字母,只有大写。
% \usepackage{bm}         % 处理数学公式中的黑斜体的宏包
% \setmainfont{LMRoman10-Regular}
% \setsansfont{LMSans10-Regular}
% \setmonofont{LMMono10-Regular}

% 英文字体设置方案二(Linux,使用自带 LM10 字体),和 LaTeX 默认字体保持一致,经典
% \usepackage{fontspec}
% \usepackage{amsmath,amssymb}
% \usepackage[CJKnumber,CJKaddspaces,CJKchecksingle,BoldFont]{xeCJK}
% \usepackage{mathrsfs}   % 一种常用于定义泛函算子的花体字母,只有大写。
% \usepackage{bm}         % 处理数学公式中的黑斜体的宏包
% \setmainfont{LMRoman10}
% \setsansfont{LMSans10}
% \setmonofont{LMMono10}

% 英文字体设置方案三(Linux,使用自带 Nimbus 字体),和 Word 模版字体保持一致,经典
% \usepackage{fontspec}
% \usepackage{mathptmx}
% \usepackage{amsmath,amssymb}
% \usepackage[CJKnumber,CJKaddspaces,CJKchecksingle,BoldFont]{xeCJK}
% \usepackage{mathrsfs}   % 一种常用于定义泛函算子的花体字母,只有大写。
% \usepackage{bm}         % 处理数学公式中的黑斜体的宏包
% \setmainfont{Nimbus Roman No9 L}
% \setsansfont{Nimbus Sans L}
% \setmonofont{Nimbus Mono L}

% 中文字体设置,使用的是 Adobe 字体,保证了在 Adobe Reader / Acrobat 下优秀的显示效果
\setCJKmainfont[BoldFont={SimSun},ItalicFont={SimSun}]{SimSun}
\setCJKsansfont{SimHei}
\setCJKmonofont{SimSun}

% 定义字体名称,可在此添加自定义的字体
\setCJKfamilyfont{song}{SimSun}
\setCJKfamilyfont{hei}{SimHei}
\setCJKfamilyfont{xhei}{STXihei} 
\setCJKfamilyfont{hwxkai}{华文行楷}
\setCJKfamilyfont{hwxhei}[AutoFakeBold = {2.17}]{华文细黑}
%\setCJKfamilyfont{fs}{Adobe Fangsong Std}
%\setCJKfamilyfont{xkai}{STXingkaiSC-Bold}

% 自动调整中英文之间的空白
% \punctstyle{quanjiao}
\XeTeXlinebreaklocale "zh"      %中文断行
\XeTeXlinebreakskip = 0pt plus 1pt %1pt左右弹性间距
% 其他字体宏包
