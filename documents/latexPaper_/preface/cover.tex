% !TEX TS-program = XeLaTeX
% !TEX encoding = UTF-8 Unicode

%%%%%%%%%%%%%%%%%%%%%%%%%%%%%%%%%%%%%%%%%%%%%%%%%%%%%%%%%%%%%%%%%%%%%% 
% 
%	大连理工大学博士论文 XeLaTeX 模版 —— 格式文件 format.tex
%	版本:0.1beta1
%	最后更新:2018.5.23
%	修改者:王天宇 (E-mail: heavenlykingwty@gmail.com)
%	编译环境:macOS MacTex XeLatex (Texpad)
% 
%%%%%%%%%%%%%%%%%%%%%%%%%%%%%%%%%%%%%%%%%%%%%%%%%%%%%%%%%%%%%%%%%%%%%% 


\cdegree{\makebox[13.25cm][s]{大连理工大学本科毕业设计(论文)}}
\ctitle{论文评审评分系统}
\etitle{Dissertation review and scoring system}

% 根据需要添加字符间距
\department{\makebox[6.1cm][c]{软件学院}}
\csubject{\makebox[6.1cm][c]{软件工程}}
\cauthor{\makebox[6.1cm][c]{刘树东}}
\cauthorno{\makebox[6.1cm][c]{201792409}}
\csupervisor{\makebox[6.1cm][c]{李明楚}}
\teacher{\makebox[6.1cm][c]{郭成}}
\cdate{\makebox[6.1cm][c]{\today}}

% 这里默认使用最后编译的时间,也可自行给定日期,注意汉字和数字之间的空格。
%\cdate{{\quad\;}\the\year~年~\the\month~月~\the\day~日}

\cabstract{

    本系统的建立旨在使用Web系统代替传统的纸质媒介对论文进行评审,使学生教师可以在网络上完成评审工作,免除了论文传递中的纸张浪费,减小了传递过程中论文丢失的可能性,方便教务统计论文评审情况,论文得分情况,常见错误出现情况。

    本文针对论文评审系统相关内容对国内外高校进行了广泛的调查,了解了学生,教师以及教务的需求,并一次为基础进行系统设计,确保最终实现的系统是师生想要的。在技术方面,本系统采用较新,稳定成熟且发展前景较好的框架搭建系统,保证系统的可维护性,例如系统的总体架构选择的前后端分离的架构设计,前端使用Vue.js框架,后端选择Spring Boot框架,且前后端可独自开发部署和升级,系统的持久化方面选择了ORM工具中的Hibernate框架,该框架可以适应多种数据库,所以校方可以方便的切换数据库类型。在界面的设计上,本系统参考了许多优秀的UI设计和交互设计,使系统界面美观且易操作。在系统功能上,提供了从上传论文到评阅论文过程中所需要的所有功能,师生可在该平台上完成整个评审任务。

}

\ckeywords{论文评审;自动化办公;毕业论文}

\eabstract{

    The establishment of this system is designed to use the Web system instead of the traditional print media to review papers, teachers can make students review the work done on the Internet, from the papers in the waste of paper, reduce the possibility of a transfer paper in the process of lost, convenient educational statistics essay review, essay scoring, common errors occur.

    This paper has carried out a wide survey of domestic and foreign colleges and universities regarding the content of paper review system, understood the needs of students, teachers and educational administration, and designed the system on the basis of a one-time, to ensure that the final system is what teachers and students want.In terms of technology, the system uses a relatively new, stable, mature and promising framework to build the system to ensure the maintainability of the system. For example, the overall architecture of the system is designed to separate the front end from the front end. The front end uses the Vue.js framework, while the back end chooses the Spring Boot framework.The persistence aspect of the system chooses the Hibernate framework in the ORM tool, which can adapt to a variety of databases, so the school can easily switch database types.In the interface design, the system refers to many excellent UI design and interaction design, so that the system interface is beautiful and easy to operate.In terms of system functions, it provides all the functions needed in the process from uploading papers to evaluating papers, and teachers and students can complete the entire evaluation task on this platform.

}

\ekeywords{Paper review; Automated office; Graduation Thesis}
\makecover