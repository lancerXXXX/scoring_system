% !TEX TS-program = XeLaTeX
% !TEX encoding = UTF-8 Unicode

\chapter*{\hfill 结  论(设计类为设计总结) \hfill}
\addcontentsline{toc}{chapter}{设计总结}
\label{conclusion}

本课题的主要内容是论文评审评分系统的研究与实现,在研究过程中分析了项目的研究背景和国内外发展情况,了解了学生,教师和教务的实际需求,以此为基础进行系统设计,详细的需求分析和系统设计保证了之后的系统实现不会出现较大的偏差。

在系统的实现过程中,系统选择了前后端分类技术体系,前端使用Vue.js框架配合vuex,vue-router,Echarts等插件和技术编写了友好的用户界面,后端使用Spring Boot框架配合安全框架Spring Security和Json Web Token实现了无状态的身份验证和授权,数据持久化方面使用Spring Data Jpa配合ORM工具Hibernate框架实现了面向对象的持久化操作方式,最后完成的系统整体安全可靠,易扩展,跨平台且易迁移。

在系统的测试过程发现了一些不足和新的想法,希望在后期的改进中实现:
\begin{enumerate}
    \item 文档的在线批阅:教师可以直接在文档上标注需要改进的地方并以评论的形式告知学生,学生同样可以在线查看文档查看问题所在。
    \item 论文感想收集:完成论文的学生可以发表感想和对晚辈的建议。
    \item 数据库优化:随着文档的增多,以往文档的管理和存储需要新的设计以保证系统的快速响应。
\end{enumerate}

当然,应该将本系统和其他系统结合起来成为整个教务系统的一部分,帮助学生更好地完成学业,帮助学校更好地管理。