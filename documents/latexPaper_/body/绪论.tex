% !TEX TS-program = XeLaTeX
% !TEX encoding = UTF-8 Unicode

\chapter{绪论}
\label{chap02}
\defaultfont
\section{研究背景}

本科生毕业论文是本科教育的最后一步,是对学生大学学习的最后一次考验,毕业论文的质量是考验本科教育质量的最后一关\cite{.2019},而毕业论文评审则是把控这最后一关的重要环节。
在十二届人大会议,李克强总理曾强调要将互联网技术与各行各业紧密相连,从而有效推动我国电子商务以及工业互联网等领域的稳健前行\cite{.2020c}。
自国家提出“互联网+”战略,互联网逐渐成为主流思路和公认的“风口”\cite{.20183}。 
但是,目前对于毕业生的论文评审还存在一些问题,最大的问题是许多学校主要还是采用纸质文档的方式,不但对于纸张是一种浪费,更是一种人力的浪费,论文的提交都需实物传递,传递时间相对于上传文档要更长,并且评审结果难以记行集中的管理\cite{.2017e},根据调查每年部分高校都会出现因为论文管理工作出现错误而导致学生毕业受阻的情况\cite{.2018},同时这种管理方式也不符合信息化办公不可阻挡的趋势。
因此,我们需要高效的论文评审系统从而可以提高毕业生的论文质量,减轻导师和学校管理的负担\cite{.2019d}。


\section{国内外发展现状}

最早的数字化校园的概念是由美国麻省理工学院在20世纪70年代提出,自此数字化校园平台、信息化教务管理平台在西方各个高校逐渐发展并走向成熟。其中高效的论文评审及管理系统在一定程度上是这些顶级高校优秀学术研究的保证。这些系统一般支持导师和学生的沟通,对学生完成情况的跟踪,作为学生安排写毕业论文的进度的参考。
随着我国社会经济以及科学技术的发展,计算机的应用也变得越来越广泛\cite{.2020h}。国内许多高校也陆续向这些国际顶级高效学习开发自己的论文管理系统,但是和国外顶级高校之间还存在着较大差距 \cite{.2017e}。虽然在国外已经有了相对成熟的范例,但是由于国情,文化等等差异,我们并不能完全照搬,仍然需要在不断的实践中逐渐摸索适合国内的数字化校园平台。就目前国内高校,从论文评审这个方面来说,一些高校尚未有论文评审系统,评审过程中仍然使用纸质作为论文的媒介,一些高校已经有了自己的在线论文评审系统,但是这些系统,虽然一定程度上实现了自动化,实际上存在着以下两个主要问题:
\begin{enumerate}
	\item 有效性问题\cite{.2019d}\\
	      部分功能并不完善,检索效率低,查找和维护困难等问题,甚至已经很老旧以至于系统设计不符合现如今的论文管理,对于部分学生,在毕业论文前期,面临着许多不同的挑战,比如应届生春招、研究生复试、公务员考试、以及应聘单位实习等,留给毕业论文的时间实际上很少\cite{.2019c},它们需要快速地了解整个进度,安排整个写毕业论文的进度,并且需要方便快速地在线提交论文以及等到反馈,因为他们可能并不在学校,但是这些系统做的并不好。
	\item 公正性问题\cite{.2019d}\\
	      公正性问题也就说评审老师的选择是否符合论文题目,如果这个选择并不合适,那么对学生论文的评判可能会有失公正\cite{Setiyani.2020}。
	      这两个关键问题处理不好可能会导致答辩人修改论文时间不足、评审人评审质量低下等问题 ,不能满足培养高素质人才的要求\cite{.2019d}。
	      基于以上讨论,选择使用B/S架构实现此系统,好处是可以屏蔽不同操作系统的差异,不必发布各个版本的桌面应用就可以投入使用。前端选择使用Vue框架,该框架具有组件化、视图,数据,结构分离、虚拟DOM、运行速度更快等优点,类似这种的前端开发框架将传统的琐碎的,杂乱无章的开发过程变成像后端一样有条理结构化的开发过程;后端选择使用Spring Boot框架,该框架具有简化配置,简化代码量等优点;数据库连接选择Hibernate框架,该框架具有轻量,移植性好等优点。
\end{enumerate}

\section{研究意义及本人工作}

本系统旨在实现论文评审评分工作的自动化,帮助学生更好地完成论文,教师更好地评审论文,教务更好得管理论文。本人主要工作为了解业务需求,定义系统功能,做好系统设计,最后将系统实现并测试系统功能。